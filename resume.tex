\documentclass[10pt,letterpaper]{article}
\usepackage[left=0.8in, right=0.8in, top=0.8in, bottom=0.9in]{geometry}
\usepackage[colorlinks,urlcolor=blue]{hyperref}
\usepackage{color}
\usepackage{enumerate}
\usepackage{fancyhdr}
\usepackage[resetlabels]{multibib}

% Define colors.
\definecolor{addresscolor}{rgb}{0.3,0.3,0.3}
\definecolor{datecolor}{rgb}{0.5,0.5,0.5}

% Turn off page numbering.
\pagestyle{fancy}
\fancyhf{}
\lfoot{\datestyle Steven K. Tjoa}
\cfoot{\datestyle \thepage}
\rfoot{\datestyle February 2018}
\renewcommand{\headrulewidth}{0.0pt}

% Set parameters and content.
\newcommand{\namestyle}{\huge \scshape}
\newcommand{\addressstyle}{\color{addresscolor} \footnotesize \rmfamily \upshape}
\newcommand{\datestyle}{\color{datecolor} \footnotesize \rmfamily \upshape}

% Remove list labels.
\renewcommand{\labelitemi}{}
\renewcommand{\labelitemii}{}

% Define bibliographies.
\newcites{j,c}{Journal Publications,Conference Publications}


\begin{document}

% Print title.
\begin{center}
    \namestyle Steven K. Tjoa \\[0.2em]
    \addressstyle Data Scientist; Machine Learning Engineer \\
    \addressstyle 19 Clementina Street, Unit 201, San Francisco, CA 94105 USA \\
    1-301-787-3693 \ $\cdot$ \ steve@stevetjoa.com \ $\cdot$ \ \url{https://stevetjoa.com}
\end{center}


\small
\section*{Highlights}

\begin{itemize}
    \item 6+ years industry experience designing and deploying machine learning models in production environments.
    \item Co-author, twelve publications in machine learning (recent: ISMIR 2017). NSF postdoctoral research fellow, 2011--13. 
    \item Sponsorship chair, ISMIR 2019. Active committee member and reviewer for international journals and conferences. Reviewed over 100 papers.
    \item Lead instructor, Workshop on Music Information Retrieval, CCRMA, Stanford, 2011--18.
    \item Creator, \url{https://musicinformationretrieval.com}, used at Stanford, Harvard, and MIT.
\end{itemize}


\section*{Featured Talks}

\begin{itemize}
    \item \href{https://youtu.be/SghMq1xBJPI}{Music Information Retrieval Using Locality Sensitive Hashing}, 9/2014. 
    \item \href{https://youtu.be/oGGVvTgHMHw}{Music Information Retrieval Using Scikit-learn}, 10/2014. 
\end{itemize}

\section*{Education}

\subparagraph{Ph. D., Electrical Engineering}
University of Maryland $\cdot$ College Park, MD $\cdot$ 1/2007--5/2011
\begin{itemize}
    \item Dissertation: ``Sparse and Nonnegative Factorizations for Music Understanding''
\end{itemize}

\subparagraph{M. S., Electrical Engineering}
University of Maryland $\cdot$ College Park, MD $\cdot$ 8/2004--12/2006
\begin{itemize}
    \item Scholarly Paper: ``Digital Image Source Coder Identification: A Non-Intrusive Forensic Methodology''
\end{itemize}

\subparagraph{B. S., Computer Engineering}
University of Maryland $\cdot$ College Park, MD $\cdot$ 8/2000--5/2004
\begin{itemize}
    \item University Honors Program.  Citation in Music Performance.
\end{itemize}


\section*{Industry Experience}

\subparagraph{Data Scientist}
Ayzenberg $\cdot$ San Francisco, CA $\cdot$ 1/2017--present
\begin{itemize}
    \item Performed statistical research into social media behavior for advertising and influencer marketing.
    \item Led development of a \textbf{bot detector} to detect bot networks in social media. Designed, implemented, evaluated, and deployed ML models using Apache Spark, AWS EMR, Redshift, Pandas, and scikit-learn.
    \item Led development of the \textbf{Social Graph}, a network of influencers on social media. Ran Neo4J graph algorithms to compute features which are consumed by other systems, e.g. influencer search engine.
    \item Managed junior engineers and interns in SF and Pasadena. Conducted one-on-one meetings. Helped with hiring decisions.
\end{itemize}


\subparagraph{Founder}
Classical Metrics, LLC $\cdot$ San Francisco, CA $\cdot$ 11/2013--12/2016
\begin{itemize}
    \item Consulted on projects in machine learning, signal processing, and data science. Provided training in data science and Python programming.
    \item Created \textbf{Violin.io}, a collection of interactive sheet music synchronized with audio recordings of classical music. Built with Django, scikit-learn, and librosa. Shut down in 2018.
\end{itemize}


\subparagraph{Research Engineer}
Humtap  $\cdot$ San Francisco, CA $\cdot$ 1/2014--1/2015
\begin{itemize}
    \item Designed, implemented, and tested signal processing and machine learning algorithms to transcribe acoustic input into symbolic musical output.
\end{itemize}


\subparagraph{Research Engineer}
Imagine Research, Inc. / iZotope, Inc. $\cdot$ San Francisco, CA $\cdot$ 6/2011--6/2013 

(iZotope acquired Imagine Research in 3/2012)
\begin{itemize}
    \item Received a two-year \textit{Small Business Postdoctoral Research Fellowship} from the U. S. National Science Foundation.
    \item Implemented algorithms to improve the MediaMined sound classifier and search engine. Examples include nonnegative matrix factorization, support vector machines, linear discriminant analysis, principal component analysis, and K-means clustering.
    \item Prepared the MediaMined production server for widespread commercial use. Technologies included Amazon Web Services (EC2, S3, RDS, SNS), PHP web server, and MySQL database server.
    \item Wrote \texttt{mediamined}, an internal Python package. Subpackages include tools for rhythmic analysis (\texttt{rhythm}), timbral analysis (\texttt{timbre}), database communication (\texttt{database}), web communication (\texttt{web}), transfers to/from Amazon S3 (\texttt{s3}), audio collection management (\texttt{collection}), unit tests (\texttt{test}), signal processing (\texttt{signal}), and machine learning (\texttt{ml}). Compliant with Sphinx for automatic HTML and PDF documentation.
    \item Wrote two automated test suites for XML validation and large-scale response-time testing.
    \item Designed evaluations and visualizations to verify the quality of MediaMined's search results.
    \item Wrote an internal dashboard for iZotope employees that displays the usage statistics of MediaMined per user and per audio collection.
\end{itemize}


\section*{Research Experience}

\subparagraph{Research Assistant, Dept. of ECE}
University of Maryland $\cdot$ College Park, MD $\cdot$ 6/2005--5/2011
\begin{itemize}
    \item Developed sparse and nonnegative factorization methods for music information retrieval (MIR) tasks such as music transcription, source separation, audio super-resolution, and instrument recognition. Techniques include nonnegative matrix factorization, orthogonal matching pursuit, K-SVD, locality sensitive hashing, approximate nearest neighbor, and multiresolution gamma filterbanks.
    \item Developed forensic methods to detect and classify traces of compression in digital images based solely upon information intrinsic to the image such as artifacts caused by pre-processing, transformation, and quantization.
    \item Received a \textit{Graduate Student Summer Research Fellowship} from the Graduate School for 2008.
\end{itemize}


% Publications.
\bibliographystylej{IEEEtran}
\bibliographystylec{IEEEtran}

\nocitej{lin-TIFS2009}
\nocitec{tjoa-ICASSP2007}
\nocitec{lin-ICME2007}
\nocitec{tjoa-ICIP2007}
\nocitec{tjoa-ICASSP2010-mult}
\nocitec{tjoa-ICASSP2010-harm}
\nocitec{stamm-ICASSP2010}
\nocitec{tjoa-ISMIR2010}
\nocitec{stamm-ICIP2010}
\nocitec{keegan2011waspaa}
\nocitec{tjoa2011ismir}
\nocitec{tsai2017ismir}

\bibliographyj{tjoa}
\bibliographyc{tjoa}



\section*{Service and Leadership Activities}

\subparagraph{Committee Member} International conferences and workshops.
\begin{itemize}
    \item \textbf{Sponsorship Chair} ISMIR 2019. Co-sponsored a successful bid to bring ISMIR 2019 to TU Delft, Netherlands.
    \item \textbf{Organizing Committee} Stanford CCRMA MIR Workshop 2011--18, ACM MIRUM 2012
    \item \textbf{Program Committee} ACM Multimedia 2013--14, AdMIRe 2012
    \item \textbf{Technical Program Committee} IEEE WIAMIS 2013; IEEE ICME 2011, 2012
\end{itemize}

\subparagraph{Reviewer} Nineteen journal and ninety-seven conference paper submissions. 
\begin{itemize}
    \item \textbf{Journals} \ 
        IEEE Trans. Audio, Speech, and Language Processing;
        IEEE Trans. Image Processing; % 1 paper
        Springer Int. J. Multimedia Information Retrieval; % 1 paper
        IEEE Trans. Information Forensics and Security; % 7 papers
        IEEE J. Selected Topics in Signal Processing; % 3 papers
        IEEE Signal Processing Letters; % 4 papers
        SPIE J. Electronic Imaging; % 1 paper
        Elsevier J. Visual Comm. and Image Representation. % 1 paper
    \item \textbf{Conferences} \ 
        IEEE ICASSP 2008, 2012, 2013, 2016, 2018; % 2 + 2 + 3 + ? + 3 papers
        IEEE WASPAA 2013, 2015, 2017; % 1 paper + ...
        ACM Multimedia 2013, 2014, 2016; % 1 paper
        ISMIR 2011--15; % 4 + 3 + 2 papers
        IEEE MMSP 2012, 2013; % 3 + 3 papers
        IEEE WIAMIS 2013; % 3 papers
        ACM MIRUM 2012; % 8 papers
        AdMIRe 2012; % 3 papers
        IEEE ICME 2011, 2012; % 15 + 7 papers
        Dagstuhl Seminar on Multimodal Music Processing 2011; % 1 paper
        IEEE ISCAS 2012; % 3 papers
        AES 42; % 1 paper
        IEEE ICIP 2008, 2010; % 5 + 5 papers
        IEEE ICCCAS 2006. % 2 papers
    \item \textit{Quality Reviewer}, ICME 2011
\end{itemize}

\subparagraph{Graduate Dean's Student Advisory Committee}
U. of Maryland $\cdot$ College Park, MD $\cdot$ 1/2008--5/2011
\begin{itemize}
    \item Advised the Graduate Dean on graduate-level academic issues at the University of Maryland including assistantship policy, mentoring, entrepreneurship, ethics, career placement, inter-departmental collaboration, awards and recognition, revision of the UMD Strategic Plan, facility renovations, and more.
\end{itemize}

\subparagraph{President, ECEGSA}
University of Maryland $\cdot$ College Park, MD $\cdot$ 7/2007--6/2008
\begin{itemize}
    \item Led the Electrical and Computer Engineering Graduate Student Association (ECEGSA), an organization that serves the 400-person ECE graduate student body through social, academic, and professional activities.  Served as the primary interface between the graduate students and faculty, staff, industry, and professional organizations.  Managed a board of nine graduate students. 
    \item Organized weekly research seminars, panel discussions, award ceremonies,  and social events.
    \item Received the \textit{Graduate Student Service Award} from the Department of Electrical and Computer Engineering for the 2007-2008 academic year.
\end{itemize}


\section*{Teaching Experience}

\subparagraph{Lead Instructor, MIR Workshop} CCRMA, Stanford University $\cdot$ Stanford, CA $\cdot$ 2011--present
\begin{itemize}
    \item Directed the annual summer workshop on music information retrieval (MIR), each lasting five full days.  Gave lectures and designed laboratory exercises on machine learning, pitch detection, chord detection, music transcription, source separation, audio alignment, genre recognition, and deep belief networks. Recruited guest lecturers from Google, Pandora, Shazam, Gracenote, Adobe, and more.
    \item Created \url{https://musicinformationretrieval.com}, a collection of educational IPython notebooks on MIR. Used at Stanford, MIT, Harvard, and International Audio Laboratories Erlangen.
\end{itemize}

\subparagraph{Instructor, MIR Workshop} CNMAT, U. C. Berkeley $\cdot$ Berkeley, CA $\cdot$ 2/2015
\begin{itemize}
    \item Taught an abbreviated version of the aforementioned Stanford CCRMA Workshop at CNMAT, Berkeley.
\end{itemize}

\subparagraph{Instructor} Galvanize, Inc. $\cdot$ San Francisco, CA $\cdot$ 2016--present
\begin{itemize}
    \item Taught courses in data science and Python programming.
\end{itemize}

\subparagraph{Lecturer, Signal and System Theory} University of Maryland $\cdot$ College Park, MD $\cdot$ Fall 2009
\begin{itemize}
    \item Taught forty-eight students. Mentored two teaching assistants. Designed lesson plans, homework assignments, and exams. Held regular office hours and review sessions.
    \item Wrote \href{http://up.stevetjoa.com/notes322_20091119.pdf}{Complementary Notes on Signals and Systems (PDF)}.
    \item \textbf{Selected Student Feedback}  ``The techniques used by Steve work very well for me.'' ``really great balance between theory and examples.'' ``good balance between theory and examples.'' ``extremely well-taught course.'' ``seemed genuinely interested in helping students.'' ``very approachable!'' ``has motivated me to branch into DSP.'' ``has been my best class this semester.'' ``Awesome class!!! my favorite this semester.''
\end{itemize}

\subparagraph{Teaching Assistant}
University of Maryland $\cdot$ College Park, MD $\cdot$ 8/2005--5/2011
\begin{itemize}
    \item \textbf{Courses} Engineering Probability Honors, Multimedia Signal Processing, Numerical Techniques in Engineering, Communications Design Laboratory. Taught over 100 students total. Designed and graded assignments, exams, and projects. Co-authored programming guides, e.g. \href{http://up.stevetjoa.com/408_0608_matlab_intro.pdf}{Introduction to Matlab Programming (PDF)}. Held regular office hours. 
    \item \textbf{Awards} Received the \textit{George Corcoran Memorial Award} from the Department of Electrical and Computer Engineering for the 2006-2007 academic year.  Received two \textit{Distinguished Teaching Assistant Awards} from the Center for Teaching Excellence.  Received a travel grant to the \textit{Lilly-East Conference on College and University Teaching}.
    \item \textbf{Selected Student Feedback}  ``Steve Tjoa - Best TA in the university.'' ``Steve was the best TA that I have ever had at UMD. Always had an answer to any question asked and used discussion time effectively.'' ``The TA Steve Tjoa, was the best TA I have ever had at Maryland'' ``he is absolutely the best TA I've ever had. His way of explaining things is really amazing.'' ``Best TA I've had in EE. Knows material well, explains concepts well.'' ``by far the best TA I ever had at UMD.''  ``great understanding of the subject.'' ``very friendly and approachable.'' ``offers insight and suggestions.'' ``always full of energy.''  ``enthusiastic.'' ``well organized, conveyed concepts clearly.'' ``understands student's needs and difficulties.'' ``fair and sincere.'' ``communicates well.'' ``one of the most dedicated TAs I've had.'' ``one of the most helpful TAs I've come across.'' ``extremely helpful at all times and motivates you to do good work in lab.'' ``holds additional office hours and is very fair at grading reports.'' ``I hope you win the best TA award.''
\end{itemize}

\end{document}
